\documentclass[a4paper, 10pt]{scrartcl}

\usepackage{ngerman}
\usepackage{amsmath}
\usepackage{amssymb}

\newcommand*\euler{\mathrm{e}}
\newcommand*\arccot{\mathrm{arccot}}

\begin{document}
\setlength{\parindent}{0em}

\section*{Mengenoperationen}
Durchschnitt: $A\cap B := \{x: x\in A \wedge x\in B\}$\\
Vereinigung: $A\cup B := \{x: x\in A \vee x\in B\}$\\
Komplement: $A\setminus B := \{x: x\in A \wedge x\notin B\}$\\

Rechenregeln für Mengen:\\
$A\cup A = A$\\
$A\cup B = B\cup A$\\
$A\cup (B\cup C) = (A\cup B)\cup C$\\
$A\cup (B\cap C) = (A\cup B)\cap (A\cup C)$\\
$(A\setminus B)\setminus C = A\setminus (B\cup C)$\\

\section*{Summation formulas and properties}
Finite sum:
\[
a_{1} + a_{2} + \dots + a_{n} = \sum_{i=1}^{n}a_{i}
\]
Infinite sum:
\[
a_{1} + a_{2} + \dots = \sum_{i=1}^{\infty}a_{i}
\]
Infinite sum as $\lim$:\\
\[
\lim_{n\to\infty}\sum_{i=1}^{n}a_{i}
\]

\section*{Geraden und Kurven}
Zwei Geraden:\\
$y = m_{1}x + c$\\
$y = m_{2}x + d$\\
stehen genau dann senkrecht aufeinander, wenn:\\
$m_{1}m_{2} = -1$\\

Allgemeine Form: $Ax + By + C = 0$, $A$ und $B$ sind nicht beide null.\\
Vertikale Gerade: $x = a$\\
Normalform: $y = mx + b$\\
Punkt-Richtungs-Form: $y - y_{0} = m(x - x_{0})$\\

Zweipunktform: $y - y_{0} = \frac{y_{1} - y_{0}}{x_{1} - x_{0}}(x - x_{0})$\\



Gleichung eines Kreises mit dem Mittelpunkt $(a, b)$
und dem Radius $r$:\\
$(x - a)^{2} + (y - b)^{2} = r^{2}$\\

\section*{Short methods of multiplication}
$(a + b)^{2} = a^{2} + 2ab + b^{2}$\\
$(a + b)^{3} = a^{3} + 3a^{2}b + 3ab^{2} + b^{3}$\\
$(a - b)^{3} = a^{3} - 3a^{2}b + 3ab^{2} - b^{3}$\\
$(a + b)(a - b) = a^{2} - b^{2}$\\
$(a + b)(a^{2} - ab + b^{2}) = a^{3} + b^{3}$\\
$(a - b)(a^{2} + ab + b^{2}) = a^{3} - b^{3}$\\
$(1 - q)(1 + q + q^{2} + \dots + q^{n}) = 1 - q^{n+1}$\\
$(1 + q)(1 - q + q^{2} - \dots (-1)^{n}q^{n}) = 1 + (-1)^{n}q^{n+1}$\\

\section*{Greatest common divisor}
$a, b, c\in \mathbb{Z}$:\\
$\gcd(a, b) = \gcd(b, a)$\\
$\gcd(\pm a, \pm b) = \gcd(b, a)$\\
$\gcd(a, a) = \vert a\vert$\\
$\gcd(a, 0) = \vert a\vert$\\
$\gcd(0, 0) = 0$\\
$\gcd(a, 1) = 1$\\
$\gcd(a, b) = \gcd(a, b\bmod a)$, $a\neq 0$\\
$\gcd(a, b + ac) = \gcd(a, b)$\\
$\gcd(a, b, c) = \gcd(a, \gcd(b, c)) = \gcd(\gcd(a, b), c)$\\
$\gcd(ac, bc) = \gcd(a, b)\cdot\vert c\vert$\\
If $a\equiv b\bmod c$: $\gcd(a, c) = \gcd(b, c)$\\
If $\gcd(a, b) = 1$: $\gcd(ab, c) = \gcd(a, c)\cdot\gcd(b, c)$\\

\section*{Trigonometrische Beziehungen}
$\sin{(\alpha + \frac{\pi}{2})} = \cos{\alpha}$\\
$\sin{(\alpha - \frac{\pi}{2})} = -\cos{\alpha}$\\
$\cos{(\alpha + \frac{\pi}{2})} = -\sin{\alpha}$\\
$\cos{(\alpha - \frac{\pi}{2})} = \sin{\alpha}$\\
$\sin{(\alpha + \pi)} = -\sin{\alpha}$\\
$\cos{(\alpha + \pi)} = -\cos{\alpha}$\\

\section*{Compound angle formulae}
$\sin{(\alpha + \beta)} = \sin{\alpha}\cos{\beta} + \cos{\alpha}\sin{\beta}$\\
$\sin{(\alpha - \beta)} = \sin{\alpha}\cos{\beta} - \cos{\alpha}\sin{\beta}$\\
$\cos{(\alpha + \beta)} = \cos{\alpha}\cos{\beta} - \sin{\alpha}\sin{\beta}$\\
$\cos{(\alpha - \beta)} = \cos{\alpha}\cos{\beta} + \sin{\alpha}\sin{\beta}$\\\\
$\tan{(\alpha + \beta)} = \frac{\tan{\alpha} + \tan{\beta}}{1 - \tan{\alpha}\tan{\beta}}$\\\\
$\tan{(\alpha - \beta)} = \frac{\tan{\alpha} - \tan{\beta}}{1 + \tan{\alpha}\tan{\beta}}$\\

\section*{Products of sines and cosines}
$\sin{\alpha}\cos{\beta} = \frac{1}{2}(\sin{(\alpha + \beta)} + \sin{(\alpha - \beta)})$\\
$\cos{\alpha}\sin{\beta} = \frac{1}{2}(\sin{(\alpha + \beta)} - \sin{(\alpha - \beta)})$\\
$\cos{\alpha}\cos{\beta} = \frac{1}{2}(\cos{(\alpha + \beta)} + \cos{(\alpha - \beta)})$\\
$\sin{\alpha}\sin{\beta} = -\frac{1}{2}(\cos{(\alpha + \beta)} - \cos{(\alpha - \beta)})$\\

\section*{Sums and differences of sines and cosines}
$\sin{\alpha} + \sin{\beta} = 2\sin{\frac{\alpha + \beta}{2}}\cos{\frac{\alpha - \beta}{2}}$\\
$\sin{\alpha} - \sin{\beta} = 2\cos{\frac{\alpha + \beta}{2}}\sin{\frac{\alpha - \beta}{2}}$\\
$\cos{\alpha} + \cos{\beta} = 2\cos{\frac{\alpha + \beta}{2}}\cos{\frac{\alpha - \beta}{2}}$\\
$\cos{\alpha} - \cos{\beta} = -2\sin{\frac{\alpha + \beta}{2}}\sin{\frac{\alpha - \beta}{2}}$\\

\section*{Double angels}

\subsection*{Double angels, $\sin{2\alpha}$}
$\sin{2\alpha} = 2\sin{\alpha}\cos{\alpha}$\\

\subsection*{Double angels, $\cos{2\alpha}$}
$\cos{2\alpha} = 2\cos^{2}{\alpha} - 1$\\
$\cos{2\alpha} = 1 - 2\sin^{2}{\alpha}$\\
$\cos{2\alpha} = \cos^{2}{\alpha} - \sin^{2}{\alpha}$\\

\subsection*{Double angels, $\tan{2\alpha}$}
$\tan{2\alpha} = \frac{2\tan{\alpha}}{1 - \tan^{2}{\alpha}}$\\

\section*{Derivatives}

Standard derivatives
$(ax^{n})' = anx^{n-1}$\\
$(\sin{ax})' = a\cos{ax}$\\
$(\sin{u})' = \cos{u}\cdot u' = u'\cos{u}$\\
$(\cos{ax})' = -a\sin{ax}$\\
$(\cos{u})' = -\sin{u}\cdot u' = -u'\sin{u}$\\
$(a^{x})' = a^{x}\cdot\ln{a}$\\
$(a^{u})' = a^{u}\cdot\ln{a}\cdot u'$\\
$(\euler^{x})' = \euler^{x}$\\
$(\euler^{ax})' = a\euler^{ax}$\\
$(\euler^{u})' = u'\euler^{u}$\\
$(\ln{x})' = \frac{1}{x}$\\
$(\ln{ax})' = \frac{1}{x}$\\
$(\ln{u})' = \frac{u'}{u}$\\
$(u^{n})' = nu^{n-1}u'$\\
$(uv)' = uv' + vu'$\\
$\left(\frac{u}{v}\right)' = \frac{vu' - uv'}{v^{2}}$\\
$\left(\sqrt{u}\right)' = \frac{u'}{2\sqrt{u}}$\\
$(\tan{u})' = \frac{u'}{\cos^{2}{u}}$\\
$(\cot{u})' = -\frac{u'}{\sin^{2}{u}}$\\
$(\arcsin{u})' = \frac{u'}{\sqrt{1 - u^{2}}}$\\
$(\arccos{u})' = -\frac{u'}{\sqrt{1 - u^{2}}}$\\
$(\arctan{u})' = \frac{u'}{1 + u^{2}}$\\
$(\arccot{u})' = -\frac{u'}{1 + u^{2}}$\\
$(\sinh{x})' = \cosh{x}$\\
$(\cosh{x})' = \sinh{x}$\\
$(\tanh{x})' = \frac{1}{\cosh^{2}{x}}$\\
$(\coth{x})' = -\frac{1}{\sinh^{2}{x}}$\\

\section*{Hyperbelfunktionen}
$\sinh{x} = \frac{\euler^{x} - \euler^{-x}}{2}$\\
$\sinh{0} = \frac{\euler^{0} - \euler^{-0}}{2} = \frac{1 - 1}{2} = 0$\\
$\cosh{x} = \frac{\euler^{x} + \euler^{-x}}{2}$\\
$\cosh{0} = \frac{\euler^{0} + \euler^{-0}}{2} = \frac{1 + 1}{2} = 1$\\
$\tanh{x} = \frac{\sinh{x}}{\cosh{x}} = \frac{\euler^{x} - \euler^{-x}}{\euler^{x} + \euler^{-x}}$\\
$\coth{x} = \frac{1}{\tanh{x}} = \frac{\cosh{x}}{\sinh{x}} = \frac{\euler^{x} + \euler^{-x}}{\euler^{x} - \euler^{-x}}$\\
$\cosh^{2}{x} - \sinh^{2}{x} = 1$\\
$\cosh^{2}{x} + \sinh^{2}{x} = \cosh{2x}$\\
$\sinh{2x} = 2\sinh{x}\cosh{x}$\\

\section*{Laplace Transform Pairs}
$F(s) = 1 \quad f(t) = \delta(t)$\\
\\
$F(s) = \euler^{-Ts} \quad f(t) = \delta(t - T)$\\
\\
$F(s) = \frac{1}{s + a} \quad f(t) = \euler^{-at}$\\
\\
$F(s) = \frac{1}{(s + a)^{n}} \quad f(t) = \frac{1}{(n - 1)!}t^{n - 1}\euler^{-at} \quad n = 1, 2, 3,\dots $\\
\\
$F(s) = \frac{1}{(s + a)(s + b)} \quad f(t) = \frac{1}{b - a}(\euler^{-at} - \euler^{-bt})$\\
\\
$F(s) = \frac{s}{(s + a)(s + b)} \quad f(t) = \frac{1}{b - a}(a\euler^{-at} - b\euler^{-bt})$\\
\\
$F(s) = \frac{s + z}{(s + a)(s + b)} \quad f(t) = \frac{1}{b - a}((z - a)\euler^{-at} - (z - b)\euler^{-bt})$\\

\end{document}
